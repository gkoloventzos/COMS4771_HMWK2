\section{Problem 1}
\subsection{A}
As perceptron is in $\mathbb{R}^d$ means that can shatter a d+1
points. As a perceptron in 2d is like a linear classifier it means
that in d dimensions can shatter d+1 points. From that I conclude that 
the VC dimension of a perceptron is d+1.
\subsection{B}
Suppose VC(H)=d. Then H will require $2^d$ distinct hypotheses to shatter d instances.
So as H can shatter in 2 classes it means that $2^d \leq \lvert H\rvert$.
\begin{eqnarray}
2^{VC(H)} \leq \lvert H\rvert \Leftrightarrow
VC(H) \leq \log_2 \lvert H\rvert
\end{eqnarray}
\subsection{C}
If we select one point then we can always shatter it.
If we select two points then we need 2 points that will have one common digit and one unique.
This is because the format of the labeling can be :\\
\begin{tabular}{ c | c | c }
  Class & ab & ac  \\
  $C_x$ & - & - \\
  $C_a$ & + & + \\
  $C_b$ & + & - \\
  $C_c$ & - & + \\
\end{tabular}
\\
So the VC dimension is at least 2.
For three numbers in order to be able to label them we must have one number common
to all three. One common with only one of the other 2. And also one unique. This creates 3 4-digit numbers.\\
\begin{tabular}{ c | c | c | c }
  Class & xyza & xybc & xzbe \\
  $C_n$ & - & - & - \\
  $C_x$ & + & + & + \\
  $C_a$ & + & - & - \\
  $C_c$ & - & + & - \\
  $C_e$ & - & - & + \\
  $C_y$ & + & + & - \\
  $C_z$ & + & - & + \\
  $C_b$ & - & + & + \\
\end{tabular}
\\
x is the common to all. \\
y is the common between 1,2. \\
z is the common between 1,3.\\
b is the common between 2,3.\\
a is unique to 1.\\
c is unique to 2.\\
e is unique to 3.\\

Also as we need 7 digit we have more than that in order to classify the all - case.
So we cannot shatter 3 points as we do not have 4-digit number.
The VC dimension is 2.
